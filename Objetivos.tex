\newpage
\chapter{Objetivos}

%%%%%%%%%%%%%%%%%%%%%%%%%%%%%%%%%%%%%%%%%%%%%%%%%%%%%%%%%%%%%%%%%%%%%%%%%%%%%%%%%%%%%%%%%%%%%

\begin{itemize}
\item \textbf{Objetivo General}

Realizar el montaje, calibración y puesta en funcionamiento de ESPECTRÓGRAFO eShel II con el que cuenta el Grupo Halley de Astronomía y Ciencias Aeroespaciales de la Universidad Industrial de Santander.\\

\item \textbf{Objetivos Específicos}
\end{itemize}

\begin{itemize}


\item Verificar el enfoque del lente colimador del espectrógrafo eShel II verificándolos con lineas de lamparas de emisión.

%Diseño y realización del montaje cámara SBIG 6303E y el espectrógrafo eShel II para verificar las lámparas de calibración Led, Tungsteno y Torio-Argón 
%Pedirle a Pisco los diseños.

\item Realizar la calibración pixel-Longitud de onda mediante el software IRAF, usando lamparas de calibración con espectros conocidos.

\item Realizar el análisis de el espectro dos estrellas de la bóveda celeste como forma de validación de la metodología.

 \item Realizar los cálculos teórico y experimentales de la masa de aire para la ciudad de Bucaramanga a partir de toma de datos con una estrella, para el análisis de la función de dispersión de punto.
%función de dispersión de punto

\end{itemize}
\newpage

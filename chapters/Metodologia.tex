
\newpage
\chapter{Metodología}


\subsection{Fase de lectura}
\noindent Para la realización de la calibración del espectrógrafo eShel II se realizara la lectura completa de los manuales correspondientes para le realización del montaje experimental, además de esto también se realizara la lectura de los manuales de los demás instrumentos a utilizar como son el telescopio CDK 17 (corrected Dall-kirkham),Cámara STXL-6303E,software IRAF (Image Reduction and Analysis Facility).\\
Estas lecturas son indispensables para la realización de cada uno de los objetivos específicos y para la correcta calibración del instrumento como objetivo general de la propuesta de grado.

\subsection{Montaje experimental  Espectrógrafo-Cámara-Telescopio.}
\noindent Luego de la adecuada lectura de cada uno de manuales de los instrumentos a utilizar se realizara el montaje experimental para la toma de datos, este consiste en la instalación del espectrógrafo con la unidad de calibración y la unión entre ellos por medio de la fibra óptica de $50 \mu m$, luego de esto se procederá a montar la cámara  SBIG STXL-6303E sobre el espectrógrafo, esta tarea es de sumo cuidado pues la alineación y el enfoque es importante para la buena captura de las imágenes, se usara para el enfoque un lente Canon EOS y se usaran múltiples acoples para relación de esta tarea.

\subsection{Adquisición de datos}
\noindent Ya con el espectrógrafo y la cámara alineados y enfocados podemos proceder a la toma de datos, esta consta en la adquisición de espectros de varias lamparas de laboratorio, esto con el fin de 

\subsection{Procesamiento de datos}

\subsection{Evaluación de los resultados y validación de la metodología}

\begin{itemize}

\item[1] Montaje del espectrógrafo con sus respectivo modulo de calibración y acople al telescopio.

\item[2] hacer el calculo de forma experimental de la rendija del espejo que permite el paso de luz al espectrógrafo.

\item[3] Realizar la captura de imágenes limpias de diferentes lamparas de emisión de laboratorio con el fin de garantizar que los elementos dispersores del espectrógrafo estén alineados con la cámara y se puedan reproducir espectros conocidos.

\item[3] Se calibrara la Montura robotizada PARAMOUNT  usando el software T-Point para garantizar un correcto apunte del telescopio al objeto de interés.






 
\end{itemize}

%%%%%%%%%%%%%%%%%%%%%%%%%%%%%%%%%%%%%%%%%%%%%%%%%%%%%%%%%%%%%%%%%%%%%%%%%%%%%%%%%%%%%%%%%%%%%%


%%%%%%%%%%%%%%%%%%%%%%%%%%%%%%%%%%%%%%%%%%%%%%%%%%%%%%%%%%%%%%%%%%%%%%%%%%%%%%%%%%%%%%%%%%%%%%
\section{Cronograma de Actividades}	
%%%%%%%%%%%%%%%%%%%%%%%%%%%%%%%%%%%%%%%%%%%%%%%%%%%%%%%%%%%%%%%%%%%%%%%%%%%%%%%%%%%%%%%%%%%%%%


\begin{center}
{\small
\begin{tabular}{|c|c|c|c|c|c|c|c|}
\hline

\textbf{Mes}/\textbf{Actividad}&\textbf{Act 1.1}&\textbf{Act 1.2}
&\textbf{Act 1.3}&\textbf{Act 2}&\textbf{Act 3}&\textbf{Act 4}&\textbf{Act 5}\\

\hline

Enero&$\bigotimes$&&&&&&\\

\hline

Febrero&$\bigotimes$&$\bigotimes$&&&&&\\

\hline

Marzo&&$\bigotimes$&$\bigotimes$&&&&\\

\hline

Abril&&&$\bigotimes$&&&&\\

\hline

Mayo&&&$\bigotimes$&$\bigotimes$&&&\\

\hline

Junio&&&$\bigotimes$&$\bigotimes$&&&\\

\hline
Julio&&&$\bigotimes$&$\bigotimes$&$\bigotimes$&&$\bigotimes$\\

\hline

Agosto&&&&&$\bigotimes$&$\bigotimes$&\\

\hline 

Septiembre&&&&&&&$\bigotimes$\\

\hline
Octubre&&&&&&&$\bigotimes$\\
\hline

\end{tabular}
}
\end{center}

\newpage
\chapter{Metodología}


\subsection{Fase de lectura}
Para la realización de la calibración del espectrógrafo eShel II se 

\subsection{Montaje experimental de espextrografo-camara-telescopio.}


\begin{itemize}

\item[1] Montaje del espectrógrafo con sus respectivo modulo de calibración y acople al telescopio.

\item[2] hacer el calculo de forma experimental de la rendija del espejo que permite el paso de luz al espectrógrafo.

\item[3] Realizar la captura de imágenes limpias de diferentes lamparas de emisión de laboratorio con el fin de garantizar que los elementos dispersores del espectrógrafo estén alineados con la cámara y se puedan reproducir espectros conocidos.

\item[3] Se calibrara la Montura robotizada PARAMOUNT  usando el software T-Point para garantizar un correcto apunte del telescopio al objeto de interés.



 
\end{itemize}

%%%%%%%%%%%%%%%%%%%%%%%%%%%%%%%%%%%%%%%%%%%%%%%%%%%%%%%%%%%%%%%%%%%%%%%%%%%%%%%%%%%%%%%%%%%%%%


%%%%%%%%%%%%%%%%%%%%%%%%%%%%%%%%%%%%%%%%%%%%%%%%%%%%%%%%%%%%%%%%%%%%%%%%%%%%%%%%%%%%%%%%%%%%%%
\section{Cronograma de Actividades}	
%%%%%%%%%%%%%%%%%%%%%%%%%%%%%%%%%%%%%%%%%%%%%%%%%%%%%%%%%%%%%%%%%%%%%%%%%%%%%%%%%%%%%%%%%%%%%%


\begin{center}
{\small
\begin{tabular}{|c|c|c|c|c|c|c|c|}
\hline

\textbf{Mes}/\textbf{Actividad}&\textbf{Act 1.1}&\textbf{Act 1.2}
&\textbf{Act 1.3}&\textbf{Act 2}&\textbf{Act 3}&\textbf{Act 4}&\textbf{Act 5}\\

\hline

Enero&$\bigotimes$&&&&&&\\

\hline

Febrero&$\bigotimes$&$\bigotimes$&&&&&\\

\hline

Marzo&&$\bigotimes$&$\bigotimes$&&&&\\

\hline

Abril&&&$\bigotimes$&&&&\\

\hline

Mayo&&&$\bigotimes$&$\bigotimes$&&&\\

\hline

Junio&&&$\bigotimes$&$\bigotimes$&&&\\

\hline
Julio&&&$\bigotimes$&$\bigotimes$&$\bigotimes$&&$\bigotimes$\\

\hline

Agosto&&&&&$\bigotimes$&$\bigotimes$&\\

\hline 

Septiembre&&&&&&&$\bigotimes$\\

\hline
Octubre&&&&&&&$\bigotimes$\\
\hline

\end{tabular}
}
\end{center}